
\section{TOP with Policy Iteration}\label{sec: TOP-PI}
The standard policy iteration (PI) algorithm is an iterative algorithm
that operates over an explicitly given MDP\cite{puterman_book_94}, and
it has two principal stages: policy evaluation and policy
improvement. At the the {\em policy evaluation} stage PI computes the
reward gained by applying currently considered policy. Specifically,
given the policy of the previous iteration, $\pi_{t-1}$, the value
function $V_t(s)$ for that policy is computed, where $V_t(s)$
represents the expected total discounted reward that can be achieved
if the environment starts at state $s$ and the agent follows
$\pi_{t-1}$. At the {\em policy improvement} stage the value function
of the current policy is used to guide the computation of the next
stage policy. Commonly, it is a policy, $\pi_{t}$, optimal with respect to the currect value function, $V_t(s)$.

Since its original introduction, both stages of the algorithm have
been refined to enable partial knowledge of the domain or introduce
conservative safety features. For instance, the value function can be
estimated, rather than computed, in environment where a direct
computation is too complex or impossible due to poor modelling by the
learner. Another possible extension is introduction of safety features
into the calculations of the new policy (e.g risk aversion). Such
modifications have been extensively used, particularly in robotics,
leading to more and more advanced methods (see
e.g.~\cite{sugiyama_et_al_2009}). Hence, by making basic PI method our
study subject in this paper, we intend to impact a large cross-cut of
applications where these learning techniques are used in a
teacher-learner (or leader-follower) setup.

In more detail, we consider the standard PI algorithm, where the
environment model is completely known and the value function is
directly computed. However, for the reasons of computational
convenience, we do introduce a modification into the policy
improvement stage. Namely, the new policy is computed as a
soft-maximisation with respect to the value function. This is done to
support differentiability of the policy with respect to the
environment dynamics\footnote{Quite commonly PI is used in combination
  with a form of neural-network computation, where soft-max is a
  natural property. As a reslt our modification does not breach the
  boundaries of the acceptable practice.}.
Formally instantiating our learner's state update $F(x_t,u)$ by PI
leads to the following set of equations:\\\ \\
{\em Policy evaluation:}\\
\centerline{
  $V_t(s)=\sum\limits_{s'}T_{u_t}(s'|s,\pi_{t-1}(s))\left[
    c(s',\pi_{t-1}(s),s)+\gamma V_t(s')
    \right]$}
{\em Policy improvement:}\\
\centerline{
$\pi_t(a|s)=\frac{1}{Z_t(s)}\exp\left(\tau_t\sum\limits_{s'}T_{u_t}(s'|s,a)\left[
    c(s',a,s)+\gamma V_t(s')
    \right]\right)$}
{\em Normalisation factor:}\\ 
\centerline{
$Z_t(s)=\sum\limits_a\exp\left(\tau_t\sum\limits_{s'}T_{u_t}(s'|s,a)\left[
    c(s',a,s)+\gamma V_t(s') \right]\right)$}

The parameter
$\tau_t$ denotes a temperature scale that shifts the soft-max towards
the greedy maximum selection. Substituting the above into the standard
TOP formulation leads to a TOP-PI optimisation problem depicted in
Figure~\ref{t_opt_PI}.
\begin{figure}[th]
\begin{tabular}{|c|} \hline \parbox{3.2 in} {\center 
$\arg\min\limits_{u_t}\sum\limits_{t=1}^{t_{max}}\sum\limits_{s,a}\pi_t(a|s)q_t(s)D^{KL}_t(s,a)$\\
$s.t.$\\
$V_t(s)=\sum\limits_{s'}T_{u_t}(s'|s,\pi_{t-1}(s))\left[
c(s',\pi_{t-1}(s),s)+\gamma V_t(s')
\right]$\\
$\pi_t(a|s)=\frac{\exp\left(\tau_t\sum\limits_{s'}T_{u_t}(s'|s,a)\left[
c(s',a,s)+\gamma V_t(s')
\right]\right)}{Z_t(s)}$\\
$Z_t(s)=\sum\limits_a\exp\left(\tau_t\sum\limits_{s'}T_{u_t}(s'|s,a)\left[
c(s',a,s)+\gamma V_t(s')
\right]\right)$\\
$\pi_0\ \ \displaystyle{given}$\\
$D^{KL}_t(s,a)=\sum\limits_{s',a'}T_{u_t}(s'|a,s)\pi_t(a'|s')\log\frac{T_{u_t}(s'|a,s)\pi_t(a'|s')}{T^0(s'|a,s)\pi^*(a'|s')}$\\
$q_t(s')=\sum\limits_s\Tilde{T}_{u_t}(s'|s)q_t(s)$\\
$\Tilde{T}_{u_t}(s'|s)=\sum\limits_aT_{u_t}(s'|s,a)\pi_t(a|s)$\\\ \\
}\\ \hline \end{tabular}
\caption{\label{t_opt_PI}TOP-PI: The complete and explicit TOP for the
  PI learner}
\end{figure}
