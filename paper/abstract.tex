\begin{abstract}

In this paper we study, for the first time explicitly, the
implications of endowing an interested party (i.e. a teacher) with the
ability to modify the underlying \emph{dynamics} of the environment,
in order to encourage an agent to learn to follow a specific
policy. We introduce a cost function which can be used by the teacher
to balance the modifications it makes to the underlying environment
dynamics, with the learner's performance compared to some ideal,
desired, policy. We formulate teacher's problem of determining optimal
environment changes as a planning and control problem, and empirically
validate the effectiveness of our model.

\end{abstract}
