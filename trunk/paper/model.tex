

%%%%%%%%%%%%%%%%%%%%%%%%%%%%%%
\section{Interaction Model}\label{sec: GeneralModel}
%%%%%%%%%%%%%%%%%%%%%%%%%%%%%%

In this section we provide a high level description of the problem and
general framework.  In the next section we provide a particular
instantiation of our framework.

For easier exposition we will present our framework to consist of a
stochastic environment and two agents, a \emph{learner} and a
\emph{teacher}, however, our formalism can be easily modified to
include an arbitrary fixed number of learning agents. The learner acts
within the environment, taking actions and receiving feedback in the
form of rewards, which depend on the action taken and the current
state of the environment in which the agent finds itself.  We assume
that the learner is rational and thus attempts to find a \emph{policy}
which describes what action to take in each environment state, so as
to maximise its expected reward.  The teacher, on the other hand,
does not act \emph{in} the environment, but rather acts \emph{on} the
environment.  In particular, the teacher has some desired
\emph{reference policy}, $\pi^*$, that it wishes the learner to
follow, but is unable to directly force the agent to take any
particular action.  Instead, it it is able to modify the environment's
\emph{dynamics} in order to cultivate the desired behaviour in the
learner.  That is, the teacher's actions are able to influence the way
the environment state changes in response to the learner's actions,
and thus influence the policy of the learner. This influence can range
from minor effects on the transition probability in a single
environment state to a principal change of the environment response
across all states and learner's actions. Dynamics modifications, or
\emph{tweaks}, come at a cost, however, and thus the goal of the
teacher is to minimise the modifications it must make to the
environment dynamics while at the same time ensuring that the policy
followed by the learner is close enough to the desired reference
policy.

We represent  the problem with the tuple  $\langle S, A, c,\gamma, U,T\rangle$ where:
\begin{itemize}
\item $S$ is the set of states, 
\item $A$ is the set of actions available to the learner,
\item  $c:S\times A\times S\rightarrow\mathbf{R}$ is the reward (or
  cost) function of the learner. $c(s',a,s)$ is the reward received by
  the learner if it has applied action $a\in A$ and the environment
  moved from state $s\in S$ to state $s'\in S$,
\item $\gamma \in (0,1)$ is a discount factor,   

\item $U$ is the set of actions (modifications to the environment)
  that the teacher can apply where $u_t\in U$ is the modification or tweak made
  at time $t$,

\item $T:S\times A\times U\rightarrow\Delta(S)$ describes the
  environment dynamics where $T_u(s'|s,a)\equiv T(s'|s,a,u)$ is the
  probability that the state will change from $s$ to $s'$ if the
  learner has applied action $a\in A$ and the teacher chose
  environment modification $u\in U$.

We assume that there exists a null modification $u^0\in U$, so that
$T^0=T_{u^0}$ are the original dynamics of the environment before any
teacher-modifications. We will use the term \emph{passive dynamics} to
refer to $T^0$ to highlight the fact that it is unchanged by the teacher.

\end{itemize}

We assume that the learner modifies and updates its policy by using
some iterative algorithm, such as value or policy iteration.  In
between iterations, the teacher is able to apply a tweak or
modification to the environment dynamics, and thus, the learner faces
a sequence of Markov Decision Problems (MDPs)~\cite{puterman_book_94},
given by tuples $<S,A,U,T_{u_t},R>$. We assume that the learner is
unaware of the teacher's actions, and thus proceeds as if the sequence
of MDPs were homogeneous.
 
 
\begin{assume}\label{assume_persistence}
At every stage the learner seeks an action policy of the form
$\pi:S\rightarrow\Delta(A)$ that would produce the highest expected
reward if $T_{u_t}$ would persist indefinitely.\footnote{This
  assumption is explicit only if the learner actually has access to
  the environment model. For most standard reinforcement learning
  algorithms this assumption would hold implicitly.}
\end{assume}


Let $x_t$ represent all information and features that the learner uses
in determining its policy, and let
$\pi_t=\pi(x_t):S\rightarrow\Delta(A)$ be the policy that corresponds
to that state.  Additionally, let $\pi^*$ be the ideal policy that the
teacher desires the learner to follow.  At time $t$, the teacher
incurs a cost, $\mathrm{Cost}(\pi_t,u_t)$, which combines the
difference between the actual policy being followed by the learner,
$\pi_t$, and the policy desired by the teacher, $\pi^*$, with the
amount of environmental modifications the teacher has had to make in
order to maintain the current dynamics, $T_{u_t}$, compared to the
initial environment dynamics, $T^0$. If we let $x_t=F(x_{t-1},u_t)$
denote one step of the learner's policy-determination algorithm and
assume that $x_0$, the original state of the learner, is known, then
it is possible to formulate the overall optimisation problem faced by
the teacher.  In particular, it is:
\begin{eqnarray*}
&\min\limits_{u_t}\sum\limits_{t=1}^{t_{max}}Cost(\pi_t,u_t)\\
&s.t.\\
&\pi_t=\pi(x_t)\\
&x_t=F(x_{t-1},u_t).
\end{eqnarray*}



%Now, denote $x_t$ the internal state of the learner's policy
%computation at iteration $t$, and let
%$\pi_t=\pi(x_t):S\rightarrow\Delta(A)$ be the policy that corresponds
%to that computation state. Also, denote $\pi^*$ the ideal reference
%action policy of the learner. Then at time $t$ the teacher incurs cost
%$Cost(\pi_t,u_t)$ that combines the distance between $\pi_t$ and
%$\pi^*$ and between $T_{u_t}$ and $T^0$. The overall optimisation
%problem for the teacher is as follows, where $x_t=F(x_{t-1},u_t)$
%denotes one step of the learner's computation algorithm and $x_0$ is
%known:
%\begin{eqnarray*}
%&\min\limits_{u_t}\sum\limits_{t=1}^{t_{max}}Cost(\pi_t,u_t)\\
%&s.t.\\
%&\pi_t=\pi(x_t)\\
%&x_t=F(x_{t-1},u_t)
%\end{eqnarray*}

This formulation of the teacher's optimisation function is fully
generic. It does not explicitly specify the learner's algorithm beyond
assuming that it is iterative. This formalism thus captures both
policy and value iteration algorithms, both with given and learned
environment models.  It can also capture the case where the learner is
capable of transfer learning~\cite{ taylor_stone_2009}. In this
scenario the learner's state $x_t$ will include structural knowledge
gathered thus far from the interaction with the environment.

Our teacher-learner interaction framework, as described, is also
generic with respect to the instantiation of the teacher's cost
function, $Cost(\pi_t,u_t)$. We argue that suitable cost functions for
this problem should incorporate information as to what environmental
modifications the teacher has performed (\emph{i.e.} the actions taken
by the teacher), along with information about how similar the policy
of the learner is to the desired policy of the teacher (\emph{i.e.}
how far the teacher is from its goal). While any function which
combines these features in a meaningful way would work, in this paper
we adopt a specific cost function derived from the
\emph{Kullback-Leibler Divergence Rate}.  We describe our proposed
cost function and the reason behind our cost function choice in more
detail in the next section.



\subsection{Teacher's Cost Computation}\label{sec:KLCost}

In this section we describe our cost function, based on the
Kullback-Leibler Divergence Rate (KL Rate or KLR).  We first provide some
background material on Kullback-Leibler Divergence (KL Divergence) and
KLR. We then describe how our cost function is defined, and provide an
argument as to why it is appropriate for our setting.


\begin{Definition}
Let $p$ and $q$ be probability distributions over some discrete random
variable. Then the Kullback-Leibler (KL) Divergence of $q$ from $p$ is
\[
D^{KL}(p||q)=\sum_i p(i)\log \frac{p(i)}{q(i)}.
\]
\end{Definition}

\noindent Informally, KL Divergence measures the difference between
two probability distributions.  KL Rate is an extension of KL
Divergence to Markov processes.
\begin{Definition}
Let $\{X^1_t\}$ and $\{X^2_t\}$ be Markov Processes.  The
Kullback-Leibler (KL) Rate is
\[
KLR(X^1||X^2)=\lim_{n\rightarrow \infty} \frac{1}{n}DL^{KL}(P(X^1=x_n)||P(X^2=x_n)).
\]
\end{Definition}
If the processes can be described by two conditional transition
matrices, $P$ and $Q$, where $P(x'|x)$ (and respectively $Q(x'|x)$) is
the probability of transitioning from state $x$ to state $x'$, then
\[
KLR(X^1||X^2)=\sum_{x} D^{KL}(P(\cdot|x)||Q(\cdot|x))p_{stat}(x)
\]
where $p_{stat}$ is the stationary distribution of
$P$~\cite{rached_alajaji_campbell_2004}.


Now, returning to our problem, let $\pi_t$ be the policy of the
learner at time $t$ and let $T_{u_t}$ be the dynamics of the
environment after the teacher has applied tweak $u_t$. If $\pi_t$ was
to be repeatedly used in the environment modified by $u_t$, this would
result in a homogeneous Markovian process defined by the transition
matrix
\[
P_t(s',a'|s,a)=T_{u_t}(s'|s,a)\pi_t(a'|s').
\]
Similarly, we can define another  Markovian process by the transition matrix
\[
P^*(s',a'|s,a)=T^0(s'|s,a)\pi^*(a'|s').
\]
This is the ideal stochastic process over state-action pairs, from the
teacher's perspective. In particular, it is formed when the teacher's
desired policy, $\pi^*$, is followed by the learner in the original,
unmodified environment. That is, the learner executes the teacher's
desired policy with no intervention from the teacher.

Assuming that $P_t$ and $P^*$ are irreducible with respect to $S\times
A$, we define our cost function as
\begin{eqnarray*}
Cost(u_t,\pi_t) & =& KLR(P_t||P^*) \\
                           &=& \sum_{s,a}D^{KL}_t(s,a)q_t(s,a)
\end{eqnarray*}
where
\[
D_t^{KL}(s,a)=D^{KL}(P_t(\cdot,\cdot|s,a)||P^*(\cdot,\cdot|s,a))
\]
and $q_t(s,a)$ is the  stationary distribution of $P_t$, so that
$q_t=P_tq_t$. 
Notably, the
stationary distribution can be decomposed (with a slight abuse of
notation) to be $q_t(s,a)=q_t(a|s)q_t(s)$ and then expressed by the
following equations:
%% \begin{eqnarray*}
%% q_t&=&q_t(a'|s')q_t(s')=P_tq_t\\
%% &=&\sum\limits_{s,a}T_{u_t}(s'|s,a)\pi_t(a'|s')q_t(a|s)q_t(s)\\
%% &=&\pi_t(a'|s')\sum\limits_sq_t(s)\sum\limits_aT_{u_t}(s'|s,a)q_t(a|s)\\
%% &&\{\displaystyle{substitute}\ \ q_t(\cdot|\cdot)\Leftarrow\pi_t(\cdot|\cdot)\}\\
%% &=&\pi_t(a'|s')\sum\limits_sq_t(s)\sum\limits_aT_{u_t}(s'|s,a)\pi_t(a|s)\\
%% &&\{\displaystyle{denote}\ \ \Tilde{T}_{u_t}(s'|s)=\sum\limits_aT_{u_t}(s'|s,a)\pi_t(a|s)\}\\
%% &=&\pi_t(a'|s')\sum\limits_sq_t(s)\Tilde{T}_{u_t}(s'|s)
%% \end{eqnarray*}
%% so that
\begin{eqnarray*}
q_t(s',a')&=&\pi_t(a'|s')q_t(s')\ \ \mbox{where}\\
q_t(s')&=&\sum\limits_s\Tilde{T}_{u_t}(s'|s)q_t(s) \ \ \mbox{and}\\
\Tilde{T}_{u_t}(s'|s)&=&\sum_{a}T_{u_t}(s'|s,a)\pi_t(a|s).
\end{eqnarray*}



Incorporating our KLR-based cost function, the overall generic Teacher
Optimisation Problem (TOP) is depicted in Figure~\ref{t_opt}. 

Notice that this formulation retains complete flexibility with respect
to the specific algorithm selected by the learner to optimise its
policy. Nevertheless, to provide further intuition and demonstrate the
feasibility of the approach, in the rest of this paper we instantiate
the algorithm $F$ to be the Policy Iteration (PI) algorithm. As would
occur with any other learning algorithm, we will identify $x_t$ with a
set of variables sufficient to capture the learner's computation state
at iteration $t$. We then will represent the change in the computation
state that occurs between two sequential iterations of the algorithm in
a functional form, $F$. The choice of PI is, therefore, not dictated
by its particularly convenient properties, but rather by the number of
applications it has been used in. As we discuss in the next section,
by using PI as our example instantiation of TOP we intend to speed up the 
dissemination and utilisation of our {\em behaviour cultivation}
teaching method to practical applications. However, before proceeding
to this particular instantiation of TOP to PI, we would like to remark
upon the meaning of the use of KLR and KL Divergence in our teaching
problem.

%%%%%%%%%%%%%%%%%
\begin{figure}[ht]
\begin{tabular}{|c|} \hline \parbox{3.2 in} {\center 
$\arg\min\limits_{u_t}\sum\limits_{t=1}^{t_{max}}\sum\limits_{s,a}\pi_t(a|s)q_t(s)D^{KL}_t(s,a)$\\
s.t.\\
$\pi_t=\pi(x_t)$\\
$x_t=F(x_{t-1},u_t)$\\
$x_0\ \ \displaystyle{is\ \ given}$\\
$D^{KL}_t(s,a)=\sum\limits_{s',a'}T_{u_t}(s'|a,s)\pi_t(a'|s')\log\frac{T_{u_t}(s'|a,s)\pi_t(a'|s')}{T^0(s'|a,s)\pi^*(a'|s')}$\\
$q_t(s')=\sum\limits_s\Tilde{T}_{u_t}(s'|s)q_t(s)$\\
$\Tilde{T}_{u_t}(s'|s)=\sum\limits_aT_{u_t}(s'|s,a)\pi_t(a|s)$\\\ \\
}\\ \hline \end{tabular}
\caption{\label{t_opt}The complete generic TOP}
\end{figure}
%%%%%%%%%%%%%%%%%%%%%%%%%%%%%%

As we have mentioned in the start of this section, KL Divergence,
$D^{KL}(P||Q)$, informally measures the difference between some
factual distribution $P$ and some desired distribution
$Q$.\footnote{Note that while KL Divergence is often called the
  \emph{distance} between two distributions, it is, in fact, not a
  true distance metric.}  As expected, $D^{KL}(P||Q)$ is minimised
exactly when $P=Q$, that is, $D^{KL}(P||P)=0$. Importantly, it
compares two distributions, rather than distribution properties, such
as the mean or variance which, for historical reasons, have been more
commonly used.\footnote{Consider, for instance, the optimality
  criteria of MDPs: the {\em expected} accumulated reward. Rather than
  being concerned with the entire shape of the obtainable reward
  distribution, it only concentrates on the expectation, necessitating
  further solution augmentation to account for requirements such as
  risk-aversion.}  In our problem, the desired distribution is
$P^*(s',a'|s,a)$ which arises if the learner follows the teacher's
desired policy with no environmental tweaks required.  By using KLR as
the cost function, we are able to assign costs to the complete variety
of possible long term deviations from $P^*$, the
ideal probability of state-action pair transition.  
These deviations
may arise from either the environmental tweak made by the teacher, or
by the learner following a non-desired policy, or some combination of
both, and thus our cost function balances these appropriately.

%Originally, $KL$ divergence (and KRL) was designed to measure the cost
%of mistaken encoding. That is, if a signal had come from a
%distribution $P$, while we have used a distribution $Q$ to encode it,
%$KL(P\|Q)$ measures the extra bits we had to
%use\cite{cover_thomas_IT_book_91}. Therefore, given the signal,
%information theory seeks to recover $Q$ to minimise the extra cost. In
%our teaching domain, the signal is the sequence of state-action pairs
%engendered by the system dynamics and the learning algorithm. However,
%by changing the system dynamics we essentially change the signal true
%distribution, rather than its encoding. Hence, we represent the ideal
%policy and the passive dynamics as the encoding distribution $Q$,
%while we modulate the signal distribution $P$ to match it.




