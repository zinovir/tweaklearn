

\section{Introduction}
%\nocite{fleming_hernandez-hernandez_CDC_97}
%\nocite{todorov_2009_framework_sup}
%\nocite{todorov_2009_framework}
%\nocite{ng_russell_2000}
%\nocite{zhang_parkes_2009_ed}
%\nocite{dufton_larson_2009}

There are three main teaching techniques applied by people: by
demonstration, by incentive, by environment dynamics
modification. While the
first two have been mapped into intelligent agent models, to the best
of our knowledge, the third one has yet to be instantiated.

Teaching by demonstration has found a significant expression in
robotics~\cite{argal_etal_2009}. However, these mostly assume that the
learner actually wishes to learn the task, as well as a certain
benevolence on behalf of the teacher with respect to the learned
task.

On the other hand, teaching by incentive has been formalised in a way
that allows to the teacher to have an interest that may contradict
that of the learner. In more detail, recently, Zhang \emph{et al}
introduced a general framework for \emph{environment
  design}~\cite{Zhang09:General}. In environment design an interested
party attempts to influence the behaviour of an agent by making limited
changes to the agent's environment. Although in general this may
include environment dynamics modification, Zhang \emph{et al} have
concentrated on teaching by incentive. In particular, Zhang \emph{et
  al} have allowed their interested party to modify the cost function
of an agent in a linear programming example~\cite{Zhang09:General}, or
to modify the rewards of an agent acting in an environment modelled as
an MDP~\cite{zhang_parkes_2008,Zhang09:Policy}.

In this paper we explicitly focus on the the implications of allowing
the interested party (teacher, in our model) to modify the
\emph{dynamics} of the environment, while leaving the reward function
of the agent alone. We first introduce a way to measure the divergence
between the realised and the passive (when no modification is applied
by the teacher) environment developments in a Markovian system. This
measure naturally incorporates and balances the teacher's effort and
the deviation of the learner's performance from an ideal reference,
that which the teacher is interested in. It also allows us to
formulate the teacher's problem as a planning and control problem, and
solve it using classical analytical and numerical tools.


While our model may be cast as an example of environment design, we
note that our instantiation differs significantly from the particular
cases studied by Zhang \emph{et al}, and therefore creates a separate
line of study. In fact, representing the teacher's task as a control
problem is far more reminiscent of the work by Banerjee and
Peng~\cite{banerjee_peng_2005}, although it too focuses on the reward
modification.

To realise the necessity and importance of teaching by
environment dynamics modification consider the following real life
scenario. A parent wishes to teach a child to ride a bicycle. The
parent may {\em demonstrate} by riding the bicycle. However, in
practice, this does not yield good results when the child attempt to
repeat the task. It is also possible to promise an {\em incentive}, be
that a candy or a trip to the movies. Unfortunately, although
increasing the child's efforts, this does not facilitate the learning
process. The most practical thing to do in this case, is to {\em
  modify the dynamics} -- add safety wheels to the bicycle. Gradually
raising the safety wheels, allows the child to accustom to the
complete range of motion possibilities and, eventually, ride an
unabridged bicycle version.

In what follows we will formally define teaching by dynamics
modification given a learning algorithm we wish to teach
(Section~\ref{sec: GeneralModel}). We will also provide a specialised
version of the formalism for a specific MDP solution technique --
Policy Iteration (PI) algorithm (Section~\ref{sec: TOP-PI}). $\{\{$
Our experiments in Section~\ref{sec: experiments} will compare the
performance of PI with and without dynamics modification. $\}\}$. We
will then conclude in Section~\ref{sec: future work} with a discussion
of further development teaching by dynamics modification.

