\documentclass{article}
\usepackage{latexsym}
\usepackage{amsfonts,amsmath,amssymb}
\usepackage{graphicx}
\usepackage{algorithmic}
\usepackage{subfigure}
\usepackage{times}

\newcommand{\ct}{\ensuremath{\mathrm{Cost}(x_t,u_t)}}
\newcommand{\cmax}{\ensuremath{\mathrm{Cost}(x_{t_{\max}},u_{t_{\max}})}}
\newcommand{\G}{\ensuremath{G(x_t,x_{t-1},u_t)}}
\newcommand{\Tu}{\ensuremath{T_{u_t}(s'|a,s)}}

\newtheorem{doubt}{Doubts and Musings}


\begin{document}

\section*{Time-limited teaching of a life-long learner}

In this scenario the learner has the opportunity to continue to learn
after the teaching process terminates. Under these circumstances the
teacher has to determine what kind of persistent environment
modification needs to be made at the end of the teaching process. The
teaching cost, as usual combined from the teaching effort and the
teaching success, needs to account for how fast and to what limit (if
at all) the learner may converge at infimum. To account for it we need to add a term, $\phi(x_{t_{\max}}, u_{t_{\max}})$, that denotes the teaching cost induced by a life-long learning process, once the active environment modifications stop. By so doing we obtain the following general form:
\[
\min \sum\limits_{t=1}^{t_{\max}}\ct + \phi(x_{t_{\max}}, u_{t_{\max}})
\]
\centerline{such that}
\begin{eqnarray*}
&x_t=F(x_{t-1},u_t)\\
\end{eqnarray*}

The key question here is how to formalise $\phi(x,u)$. At time $t\leq
t_{\max}$ the cost was calculated by posing the question ``how would
the system progress (compared to the ideal), if {\em both} the
teaching and learning stopped at that point''. However, the terminal
cost $\phi(x,u)$ needs to account for potentially unstable development
of the learner. In particular, if the learner does not converge it
poses the question of predicting the set of ``knowledge points''
$x_{t>t_{\max}}$ that the learner will visit, and how they will effect
teacher's costs. It can still be expressed by a KLR, and for many
systems calculated or bound (see
e.g. ~\cite{sun_mehta_2010,do_2003,zuk_2006,vidyasagar_2007}), but we
also must remember that there are cases where KLR does not
exist~\cite{shields_93}. Although it is unclear how common such cases
are in practical, and especially engineering, applications, it is
necessary to keep in mind the potential need for divergence measures
approximating KLR (e.g.~\cite{rached_alajaji_campbell_2001}) or
different from it completely.

Now, a prominent feature in making KLR computationally feasible
measure is the stationarity of the two compared processes, or the
process's stabilisation in this sense (aka {\em mixing} property). In
terms of our learner's progress such a property is produced by a
learner that is, in fact, capable of finding a (locally)
learner-optimal policy in a given environment. In turn, be able to
control the teaching cost of this type of a life-long learner, it is {\em
  sufficient} (but not {\em necessary}) for the teacher to guarantee
the following two conditions:
\begin{itemize}
\item Exists $x^*$ so that $x^*=F(x^*,u_T)$. That is the learner has a stable knowledge state in the final environment. 
\item $F$ is a contraction mapping on $B_r(x^*)$, where $r\geq\|x^*-x_{T-1}\|)$. That is the last controlled knowledge point $x_{T-1}$ is in a small sphere around $x^*$, where the learning process strongly converges. 
\end{itemize}

The finite-time teaching problem of a life-long learner, therefore becomes:

\[
\min \sum\limits_{t=1}^{t_{\max}}\ct + \widehat{\phi}(x^*, u_{t_{\max}})
\]
\centerline{such that}
\begin{eqnarray*}
&x_t=F(x_{t-1},u_t)\\
&x^*=F(x^*,u_{t_{\max}})\\
&r=\|x_{t_{\max}}-x^*\|\\
&\forall x,y\in B_r(x^*),\ \ \|F(x,u_{t_{\max}})-F(y,u_{t_{\max}})\|<\|x-y\|\\
\end{eqnarray*}

\begin{doubt}
Although it is true that $x^*$ is a convergence point, the rate of
convergence may be important from the cost's point of
view. Discrepancy along the convergence path may accumulate to a
significant amount independent from $x^*$ itself. In this case, the
optimisation function becomes again $\phi(x_{t_{\max}},
u_{t_{\max}})$, but with the additional benefit of new constraints,
that may focus the search for teacher's solution. Nonetheless, we may
consider event a stronger requirement:
$\|F(x,u_{t_{\max}})-F(y,u_{t_{\max}})\|\geq\gamma\|x-y\|$, where
$\gamma<1$. There are also many other weaker convergence
characteristics that we can consider, after all exponential collapse
is a {\em very} strong demand.
\end{doubt}

\begin{doubt}
It may not be possible to determine the equilibrium point $x^*$
efficiently. Nonetheless, if $F$ is sufficiently accomodating we may
be able to find a bound on large deviations (something like the law of large number, e.g.~\cite{liu_liu_94}), similarly to what happens
with large deviations of colouring number. There are also some
theories concerning the long-term behaviour of time non-homogeneous
Markov chains, but they are core math, it will take much more effort
to tie it in with them.
\end{doubt}

\section*{The case of MDP and PI}

PI has good convergence rates even if the value function is
approximated (see e.g.~\cite{bertsekas_book_adp}. We can also assume
that it is capable of handling multiple argodic blocks, inferring
relevance from the initial state distribution. As a result we can drop
the requirement for the update being a contraction. Furthermore, since
the environment in the $\phi$ term is fixed, and the learning process
converges quickly, we can assume that contribution of non-homogeneous
transitions is negligible. In other words, we can consider $\phi$ as
the measure of long term discrepancy between the converged learner's
behaviour and the ideally controlled initial environment. Since this
is fully consistent with the meaning of
$\mathrm{Cost}(x^*,u_{t_{\max}})$, we can use this expression in place
of $\phi$. This leads to the following optimisation problem, where $F$
denotes the PI update:

\[
\min \sum\limits_{t=1}^{t_{\max}}\ct + \mathrm{Cost}(x^*, u_{t_{\max}})
\]
\centerline{such that}
\begin{eqnarray*}
&x_t=F(x_{t-1},u_t)\\
&x^*=F(x^*,u_{t_{\max}})\\
\end{eqnarray*}

We can rewrite as before, and using KKT condition (although I'm not quite locked on this) to reuse to maximum our previous calculations:

\[
\min \sum_{t=1}^{t_{\max}} \ct + \mathrm{Cost}(x^*, u_{t_{\max}})
\]
\centerline{such that}
\begin{eqnarray*}
&G(x_t,x_{t-1},u_t)=0\\
&\frac{\partial}{\partial x}G(x,x,u_{t_{\max}})|_{x=x^*}=0
\end{eqnarray*}
where
\[
x_t=\{\pi_t(a|s),q_t(s),V_{t}(s),z_t(s)\} \mbox{ and $u_t$ is the perturbation.}
\]

\begin{doubt}
There's a nice section of control approaches dedicated to this type of
problems. That is, move the system through a critical region to
``safety'', then switch to standard controller to finish stabilisation
(starting with e.g.~\cite{yanushevsky_91}, but the list is long). The
choice of ``safe'' region is also part of the problem. So we need to
decide if we are going to bother with our own solution for the
discrete case, or go with approximate PI continuous state and plug it
into one of those control methods. Although I haven't found an
implementation on-line yet, and the derivatives calculations will
still be required.
\end{doubt}

\bibliography{lifelong}
\bibliographystyle{plain}

\end{document}
